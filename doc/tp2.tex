% FACULTAD DE INGENIERÍA, UNIVERSIDAD DE BUENOS AIRES
%
% 75.10 - Técnicas de Diseño
% Trabajo práctico nro 2
%
% INFORME

\documentclass[12pt]{article}

\usepackage[a4paper,headheight=16pt,scale={0.7,0.8},hoffset=0.5cm]{geometry}
\usepackage[spanish]{babel}
\usepackage{graphicx}
\usepackage[utf8]{inputenc}

% Para poner el texto "Figura X" en negrita:
\usepackage[hang,bf]{caption}

% Símbolos varios.
\usepackage{textcomp}

\title{Técnicas de Diseño: TP2}
\author{Barrios, Federico; Bosch, Florencia; Navarro, Patricio}

%------------------------- Inicio del documento ---------------------------
\begin{document}

\begin{center}
\vspace*{7 cm}
\textsc{\LARGE Universidad de Buenos Aires}\\[0.3cm]
\textsc{\LARGE Facultad de Ingeniería}\\[1.2cm]
\textsc{\Large 75.10 -- Técnicas de Diseño}\\[0.3cm]
\textsc{\Large Trabajo Práctico 2}\\[1.2cm]
\end{center}

\begin{flushright}
{\large
Grupo 3:\\[0.1cm]
Barrios, Federico -- 91954\\
Bosch, Florencia -- 91867\\
Navarro, Patricio -- 90007\\[0.4cm]
$2^{do}$ cuatrimestre de 2013}
\end{flushright}

\thispagestyle{empty}

\newpage

% Pongo el índice en una página aparte:
\tableofcontents
% Hago que las páginas se comiencen a contar a partir de aquí:
\setcounter{page}{1}
\newpage

\section{Especificación}
Se propone en esta instancia seguir agregando funcionalidades al framework de testing implementado,
con la diferencia de que esta vez desarrollamos sobre el trabajo de otro grupo. 

Los requerimientos 
pedidos por la cátedra son ahora un límite de tiempo de corrida, de manera de registrar como fallos
casos de prueba cuya ejecución exceda un tiempo dado. Además se pide una función de \textit{memoria}
para recordar aquéllos casos de pruebas que hayan terminado exitosamente, para así ejecutar en una
corrida posterior los fallados, los que devienen en error y los que se agregan. Es necesario ofrecer
dos tipos de \textit{Store}, y preferentemente una manera sencilla para que el usuario pueda agregar más.

\section{Análisis de la implementación del grupo 2}
Antes de empezar con la implementación de las funcionalidades nuevas, el grupo se propuso entender el 
funcionamiento del framework.

\subsection{Problemas encontrados}
Se identificaron los siguientes problemas:

\begin{itemize}
	\item \textbf{Código con errores:} pudimos identificar varios errores en el trabajo que tuvimos que
		corregir antes de empezar a implementar:
		
		\begin{itemize}
			\item El código contenía warnings referidos a un mal uso de la genericidad de Java.
			
			\item Maven no compilaba debido a la omisión de la dependendencia que incluye el paquete 
				com.sun.javaws.
				
		\end{itemize}
		
		Estos dos errores fueron los primeros en ser solucionados.
		
	\item \textbf{Documentación escasa:} 
		
	\item \textbf{Errores funcionales:} se detectaron varias situaciones en dónde el comportamiento
		no es el esperado de un framework de testing:
		
		\begin{itemize}
			\item El contexto del fixture puede ser modificado por cada caso de prueba, acarreando las
				modificaciones y derivando en un mal funcionamiento del resto de los tests.
				El comportamiento esperado era que el contexto no fuese modificado y que el
				resultado de la corrida sea independiente del orden de ejecución de los tests.

			\item También relacionado con lo anterior, el contexto no permitía almacenar objetos de cualquier 
				tipo, que es lo esperado de un framework.
				
			\item El resultado de la corrida en formato XML no respeta el esquema modelo brindado por 
				el curso.
				
			\item Las corridas de cada test figuraban con un tiempo de ejecución diferente en cada
				tipo de reporte. Esto se debía a que se detenía el temporizador en la función de
				escritura.

		\end{itemize}	
		
	\item \textbf{Violación de los principios de diseño:} nos resultó muy complicada la tarea de implementar 
		las funcionalidades nuevas debido a que el diseño no respetaba el principio de clausura ante 
		cambios y apertura ante modificaciones; imposibilitando, incluso, la adición de ciertas pruebas. 
		Además, la clase TestReport tiene muchas responsabilidades.	

	\item \textbf{Uso de herramientas tecnológicas:} en la clase TestSuite, a la hora de exportar
		el reporte se usa reflexión para diferenciar el tipo del Test pasado por parámetro,
		pudiendo ser alguna de las dos clases que heredan de ella: TestCase o TestSuite.
		Vemos esto como una violación grave del paradigma de orientación a objetos y de la
		consigna del enunciado.
	
	\item \textbf{Poca cantidad de pruebas:} notamos que el trabajo tiene muy pocas pruebas
		implementadas, resultando dificultosa la tarea de añadir funcionalidades y
		de refactorizar, dado que no podemos asegurar el funcionamiento correcto del trabajo
		después de haber modificado o agregado código.
		
	\item \textbf{Estilo:} muchas variables no respetan la convención \textit{camelCase} 
		de Java; además, el estilo de ubicación de llaves y demás no es coherente dentro del
		código.
		
\end{itemize}


\begin{figure}[h!]
\begin{center}
	\includegraphics[scale=0.50,angle=90]{./ClassDiagram3}
\end{center}
	\caption{diagrama de clases del trabajo práctico (entrega 3).}
\end{figure}

	
\section{Diseño}

\subsection{Responsabilidades de las clases agregadas}
Se describe a continuación la responsabilidad de cada clase nueva:
\begin{itemize}
\end{itemize}

\section{Pruebas}
Blah
	
\section{Conclusiones}
Durante el desarrollo del trabajo práctico nos tuvimos que enfrentar con
código no escrito por nosotros 

importancia del código bien documentado, documentación actualizada, importancia del principio abierto/cerrado, importancia de un
set de pruebas unitarias, código legacy por primera vez
\end{document}
