% FACULTAD DE INGENIERÍA, UNIVERSIDAD DE BUENOS AIRES
%
% 75.10 - Técnicas de Diseño
% Trabajo práctico nro 2
%
% INFORME

\documentclass[12pt]{article}

\usepackage[a4paper,headheight=16pt,scale={0.7,0.8},hoffset=0.5cm]{geometry}
\usepackage[spanish]{babel}
\usepackage{graphicx}
\usepackage[utf8]{inputenc}

% Para poner el texto "Figura X" en negrita:
\usepackage[hang,bf]{caption}

% Símbolos varios.
\usepackage{textcomp}

\title{Técnicas de Diseño: TP2}
\author{Barrios, Federico; Bosch, Florencia; Navarro, Patricio}

%------------------------- Inicio del documento ---------------------------
\begin{document}

\begin{center}
\vspace*{7 cm}
\textsc{\LARGE Universidad de Buenos Aires}\\[0.3cm]
\textsc{\LARGE Facultad de Ingeniería}\\[1.2cm]
\textsc{\Large 75.10 -- Técnicas de Diseño}\\[0.3cm]
\textsc{\Large Trabajo Práctico 2}\\[1.2cm]
\end{center}

\begin{flushright}
{\large
Grupo 3:\\[0.1cm]
Barrios, Federico -- 91954\\
Bosch, Florencia -- 91867\\
Navarro, Patricio -- 90007\\[0.4cm]
$2^{do}$ cuatrimestre de 2013}
\end{flushright}

\thispagestyle{empty}

\newpage

% Pongo el índice en una página aparte:
\tableofcontents
% Hago que las páginas se comiencen a contar a partir de aquí:
\setcounter{page}{1}
\newpage

\section{Especificación}
Se propone en esta instancia seguir agregando funcionalidades al framework de testing implementado,
con la diferencia de que esta vez desarrollamos sobre el trabajo de otro grupo. 

Los requerimientos 
pedidos por la cátedra son ahora un límite de tiempo de corrida, de manera de registrar como fallos
casos de prueba cuya ejecución exceda un tiempo dado. Además se pide una función de \textit{memoria}
para recordar aquéllos casos de pruebas que hayan terminado exitosamente, para así ejecutar en una
corrida posterior los fallados, los que devienen en error y los que se agregan.

\section{Análisis}
Antes de empezar con la implementación, el grupo se propuso entender el 
funcionamiento del framework 

nice to have: diagramas de clases inicial y final.
\begin{figure}[h!]
\begin{center}
	\includegraphics[scale=0.50,angle=90]{./ClassDiagram3}
\end{center}
	\caption{diagrama de clases del trabajo práctico (entrega 3).}
\end{figure}

	
\section{Diseño}

\subsection{Responsabilidades de las clases agregadas}
Se describe a continuación la responsabilidad de cada clase nueva:
\begin{itemize}
\end{itemize}

\section{Pruebas}
Se incluyen con la implementación varios sets de pruebas:

	\begin{enumerate}
	\item Pruebas unitarias: estas pruebas verifican el funcionamiento de 
	las clases desarrolladas usando jUnit 4. 
	
	Se intentó alcanzar una cobertura del 100\% del código con pruebas 
	unitarias, sin embargo se notó que había varias clases muy simples como 
	TestResult que probarlas pareció inútil.

	\item Pruebas de entorno: la intención de estas pruebas fue probar una 
	clase externa como lo haría el usuario.
	Se creo una clase con un comportamiento trivial y se generaron casos de
	prueba para después correrlos. 
	
	Se insertaron tanto pruebas que corren correctamente como pruebas que
	fallan, de manera de mostrar el funcionamiento completo del entorno.
	
	El mismo de set de pruebas se corrió con jUnit 4 para mostrar que 
	efectivamente los resultados obtenidos son equivalentes.
		\begin{enumerate}
		\item Pruebas de comparación: que verifican que funcione 
		correctamente las aserciones del framework ante diferentes 
		situaciones.
		\item Pruebas de anidación: que verifican que se ejecuten todos
		los casos de prueba en una corrida, anidando varios niveles
		de suites con casos.
		\item Pruebas de expresiones regulares: que verifican que se
		respete que sólo se corran aquéllas pruebas que cumplan con
		la expresión regular indicada.
		\item Pruebas de fixture: que verifican que se respeten los setUps
		y los tearDowns que corresponden para cada corrida.
		\end{enumerate}
	\end{enumerate}

	Se realizaron pruebas sobre las nuevas funcionalidades del framework, como
	 es la ejecución de tests con filtro de TAGS y nombre que cumplan con una expresión 
	regular, tests que poseen el TAG SKIP, y casos de pruebas verificando el tiempo de
	 los tests. Estos casos de prueba se encuentran en la clase FrameworkTest, intentando
	 alcanzar en la misma una cobertura del 100\% del framework.
	
\section{Conclusiones}
Durante el desarrollo del trabajo pudimos aplicar los conceptos adquiridos a lo
largo de la materia para hacer un código más mantenible, más legible y más
estandarizado que el que hubiéramos escrito antes de cursar.

Pudimos verificar que el diseño que implementamos en la primera entrega estaba
abierto a modificaciones y cerrado ante cambios porque las extensiones de los
requerimientos las pudimos hacer casi sin tocar código existente.

También hicimos uso extensivo de las herramientas que nos ofrece el IDE según lo
visto en las clases prácticas, como por ejemplo a la hora de refactorizar.

Finalmente pudimos apreciar las ventajas de tener desde el primer momento un set
de pruebas confiable, pues pudimos estar seguros en todo momento de que las 
modificaciones que introducíamos no hacían que código antiguo dejara de funcionar.

\end{document}
