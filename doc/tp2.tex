% FACULTAD DE INGENIERÍA, UNIVERSIDAD DE BUENOS AIRES
%
% 75.10 - Técnicas de Diseño
% Trabajo práctico nro 2
%
% INFORME

\documentclass[12pt]{article}

\usepackage[a4paper,headheight=16pt,scale={0.7,0.8},hoffset=0.5cm]{geometry}
\usepackage[spanish]{babel}
\usepackage{graphicx}
\usepackage[utf8]{inputenc}

% Para poner el texto "Figura X" en negrita:
\usepackage[hang,bf]{caption}

% Símbolos varios.
\usepackage{textcomp}

\title{Técnicas de Diseño: TP2}
\author{Barrios, Federico; Bosch, Florencia; Navarro, Patricio}

%------------------------- Inicio del documento ---------------------------
\begin{document}

\begin{center}
\vspace*{7 cm}
\textsc{\LARGE Universidad de Buenos Aires}\\[0.3cm]
\textsc{\LARGE Facultad de Ingeniería}\\[1.2cm]
\textsc{\Large 75.10 -- Técnicas de Diseño}\\[0.3cm]
\textsc{\Large Trabajo Práctico 2}\\[1.2cm]
\end{center}

\begin{flushright}
{\large
Grupo 3:\\[0.1cm]
Barrios, Federico -- 91954\\
Bosch, Florencia -- 91867\\
Navarro, Patricio -- 90007\\[0.4cm]
$2^{do}$ cuatrimestre de 2013}
\end{flushright}

\thispagestyle{empty}

\newpage

% Pongo el índice en una página aparte:
\tableofcontents
% Hago que las páginas se comiencen a contar a partir de aquí:
\setcounter{page}{1}
\newpage

\section{Especificación}
Se propone en esta instancia seguir agregando funcionalidades al framework de testing implementado,
con la diferencia de que esta vez desarrollamos sobre el trabajo de otro grupo. 

Los requerimientos 
pedidos por la cátedra son ahora un límite de tiempo de corrida, de manera de registrar como fallos
casos de prueba cuya ejecución exceda un tiempo dado. Además se pide una función de \textit{memoria}
para recordar aquéllos casos de pruebas que hayan terminado exitosamente, para así ejecutar en una
corrida posterior los fallados, los que devienen en error y los que se agregan. Es necesario ofrecer
dos tipos de \textit{Store}, y preferentemente una manera sencilla para que el usuario pueda agregar más.

\section{Análisis de la implementación del grupo 2}
Antes de empezar con la implementación de las funcionalidades nuevas, el grupo se propuso entender el 
funcionamiento del framework.

\subsection{Problemas encontrados}
Se identificaron los siguientes problemas:

\begin{itemize}
	\item \textbf{Uso de herramientas tecnológicas:} en la clase TestSuite, a la hora de exportar
		el reporte se usa reflexión para diferenciar el tipo del Test pasado por parámetro,
		pudiendo ser alguna de las dos clases que heredan de ella: TestCase o TestSuite.
		Vemos esto como una violación grave del paradigma de orientación a objetos y de la
		consigna del enunciado.
	
	\item \textbf{Cantidad escasa de pruebas:} notamos que el trabajo tiene muy pocas pruebas
		implementadas, resultando dificultosa la tarea de añadir funcionalidades y
		de refactorizar, dado que no podemos asegurar el funcionamiento correcto del trabajo
		después de haber modificado o agregado código.		 

	\item \textbf{Estilo:} muchas variables no respetan la convención \textit{camelCase} 
		de Java, y el estilo de ubicación de llaves y demás no es coherente en el
		código.
		
\end{itemize}


nice to have: diagramas de clases inicial y final.
\begin{figure}[h!]
\begin{center}
	\includegraphics[scale=0.50,angle=90]{./ClassDiagram3}
\end{center}
	\caption{diagrama de clases del trabajo práctico (entrega 3).}
\end{figure}

	
\section{Diseño}

\subsection{Responsabilidades de las clases agregadas}
Se describe a continuación la responsabilidad de cada clase nueva:
\begin{itemize}
\end{itemize}

\section{Pruebas}
Blah
	
\section{Conclusiones}
importancia del código bien documentado, documentación actualizada, importancia del principio abierto/cerrado, importancia de un
set de pruebas unitarias, código legacy por primera vez
\end{document}
